%%%%%%%%%%%%%%%%%%%%%%%%%%%%%%%%%%%%%%%%%%%%%%%
%CODE LISTING SYNTAX COLORING
%Source code listings from files and present them using Sublime Text
%syntax coloring.

%CALL LISTING PACKAGE IN SEPARATE BEAMER.TEX WITH:
%\newcommand\whiteback{0} %code listing background color (0: black, 1: white)
%%%%%%%%%%%%%%%%%%%%%%%%%%%%%%%%%%%%%%%%%%%%%%%%
%CODE LISTING SYNTAX COLORING
%Source code listings from files and present them using Sublime Text
%syntax coloring.

%CALL LISTING PACKAGE IN SEPARATE BEAMER.TEX WITH:
%\newcommand\whiteback{0} %code listing background color (0: black, 1: white)
%%%%%%%%%%%%%%%%%%%%%%%%%%%%%%%%%%%%%%%%%%%%%%%%
%CODE LISTING SYNTAX COLORING
%Source code listings from files and present them using Sublime Text
%syntax coloring.

%CALL LISTING PACKAGE IN SEPARATE BEAMER.TEX WITH:
%\newcommand\whiteback{0} %code listing background color (0: black, 1: white)
%%%%%%%%%%%%%%%%%%%%%%%%%%%%%%%%%%%%%%%%%%%%%%%%
%CODE LISTING SYNTAX COLORING
%Source code listings from files and present them using Sublime Text
%syntax coloring.

%CALL LISTING PACKAGE IN SEPARATE BEAMER.TEX WITH:
%\newcommand\whiteback{0} %code listing background color (0: black, 1: white)
%\input{\string~/lib/beamer/codelisting}


\usepackage{color}     %make custom colors for syntax coloring
\usepackage{listings}  %allows code listings
\usepackage{textcomp}  %allows apostrophes to be straight (unidirectional)
% \usepackage{framed}
% \usepackage{caption}
\usepackage{bm}
\usepackage{tcolorbox}        %use for code listing color definitions
\tcbuselibrary{listings}      %allow color defenitions in code listings

\captionsetup[lstlisting]{font={small,tt}}

%Custom Colors
\definecolor{mygreen}{rgb}{0,0.6,0}
\definecolor{mygray}{rgb}{0.5,0.5,0.5}
\definecolor{mymauve}{rgb}{0.58,0,0.82}

\definecolor{sublimeblack}{HTML}{272822}
\definecolor{sublimered}{HTML}{F92672}
\definecolor{sublimeblue}{HTML}{66D9EF}
\definecolor{sublimeyellow}{HTML}{E6DB74}
\definecolor{sublimegrey}{HTML}{75715E}
\definecolor{sublimegreen}{HTML}{66CC33}
\definecolor{sublimeorange}{HTML}{FD971F}
\definecolor{sublimepurple}{HTML}{AE81FF}


%White or Black background toggle
% \newcommand\whiteback{0}
\ifnum\whiteback=1%
  %White Background
  \newcommand\backclr{white}
  \newcommand\txtclr{\color{black}}
  \newcommand\comclr{\color{mygreen}}
  \newcommand\keyclr{\color{blue}}
  \newcommand\ndkeyclr{\color{sublimered}}
  \newcommand\strclr{\color{mymauve}}
  \newcommand\frameon{single} %single-frame box around code
\else
  %Black Background (like SublimeText)
  \newcommand\backclr{sublimeblack}
  \newcommand\txtclr{\color{white}}
  \newcommand\comclr{\color{sublimegrey}}
  \newcommand\keyclr{\color{sublimeblue}}
  \newcommand\ndkeyclr{\color{sublimered}}
  \newcommand\strclr{\color{sublimeyellow}}
  \newcommand\frameon{false} %no frame for black background
\fi


%Listing Style
\lstdefinestyle{mystyle}{%
  % backgroundcolor=\backclr,   % choose the background color; you must add \usepackage{color} or \usepackage{xcolor}
  basicstyle=\ttfamily\txtclr\scriptsize, % the SIZE OF THE FONTS that are used for the code
  breakatwhitespace=false,         % sets if automatic breaks should only happen at whitespace
  % breaklines=true,                 % sets automatic line breaking
  captionpos=b,                    % sets the caption-position to bottom
  commentstyle=\comclr,            % comment color
  deletekeywords={...},            % if you want to delete keywords from the given language
  escapeinside={\%*}{*)},          % if you want to add LaTeX within your code
  extendedchars=true,              % lets you use non-ASCII characters; for 8-bits encodings only, does not work with UTF-8
  frame=\frameon,                  % adds a frame around the code
  keepspaces=true,                 % keeps spaces in text, useful for keeping indentation of code (possibly needs columns=flexible)
  columns=flexible,
  otherkeywords={zip,enumerate,True,False,None,...},  %add words to be highlighted
  keywordstyle=\keyclr,            % keyword (e.g. 'print') color
  language=Python,                 % the language of the code
  upquote=true,                    % make apostrophes straight (unidirectional)
  alsoletter={<>=-+*/!},            % to avoid coloring operators when they're not
  ndkeywords={=,+,-,*,**,/,+=,*=,-=,/=,<=,>=,==,!=,<,>,... },  % operator keywords
  ndkeywordstyle=\ndkeyclr,         % style of operator keywords
  numbers=left,                    % where to put the line-numbers; possible values are (none, left, right)
  numbersep=5pt,                   % how far the line-numbers are from the code
  numberstyle=\tiny\color{mygray}, % line-number style: size, color
  rulecolor=\color{black},         % if not set, the frame-color may be changed on line-breaks within not-black text (e.g. comments (green here))
  showspaces=false,                % show spaces everywhere adding particular underscores; it overrides 'showstringspaces'
  showstringspaces=false,          % underline spaces within strings only
  showtabs=false,                  % show tabs within strings adding particular underscores
  stepnumber=1,                    % the step between two line-numbers. If it's 1, each line will be numbered
  stringstyle=\strclr,             % string color
  tabsize=4,                       % sets default tabsize
}


%Function that calls listing of specific file with custom coloring
  %first input is filename to list
  %next two inputs are start and end line numbers
    %(use 0 and large number for all lines)
\newcommand\mylisting[3]{
  \tcbinputlisting{
        listing file=#1,
        arc=0pt,
        top=0mm,
        bottom=0mm,
        left=0mm,
        right=0mm,
        boxrule=0pt,
        colback=\backclr,
        listing only,
        listing options={
                            style=mystyle,            %use custom style
                            firstline=#2,lastline=#3, %lines of code to use
                            firstnumber=#2,           %same numbering as script
                        },
        hbox
      }
      }



\usepackage{color}     %make custom colors for syntax coloring
\usepackage{listings}  %allows code listings
\usepackage{textcomp}  %allows apostrophes to be straight (unidirectional)
% \usepackage{framed}
% \usepackage{caption}
\usepackage{bm}
\usepackage{tcolorbox}        %use for code listing color definitions
\tcbuselibrary{listings}      %allow color defenitions in code listings

\captionsetup[lstlisting]{font={small,tt}}

%Custom Colors
\definecolor{mygreen}{rgb}{0,0.6,0}
\definecolor{mygray}{rgb}{0.5,0.5,0.5}
\definecolor{mymauve}{rgb}{0.58,0,0.82}

\definecolor{sublimeblack}{HTML}{272822}
\definecolor{sublimered}{HTML}{F92672}
\definecolor{sublimeblue}{HTML}{66D9EF}
\definecolor{sublimeyellow}{HTML}{E6DB74}
\definecolor{sublimegrey}{HTML}{75715E}
\definecolor{sublimegreen}{HTML}{66CC33}
\definecolor{sublimeorange}{HTML}{FD971F}
\definecolor{sublimepurple}{HTML}{AE81FF}


%White or Black background toggle
% \newcommand\whiteback{0}
\ifnum\whiteback=1%
  %White Background
  \newcommand\backclr{white}
  \newcommand\txtclr{\color{black}}
  \newcommand\comclr{\color{mygreen}}
  \newcommand\keyclr{\color{blue}}
  \newcommand\ndkeyclr{\color{sublimered}}
  \newcommand\strclr{\color{mymauve}}
  \newcommand\frameon{single} %single-frame box around code
\else
  %Black Background (like SublimeText)
  \newcommand\backclr{sublimeblack}
  \newcommand\txtclr{\color{white}}
  \newcommand\comclr{\color{sublimegrey}}
  \newcommand\keyclr{\color{sublimeblue}}
  \newcommand\ndkeyclr{\color{sublimered}}
  \newcommand\strclr{\color{sublimeyellow}}
  \newcommand\frameon{false} %no frame for black background
\fi


%Listing Style
\lstdefinestyle{mystyle}{%
  % backgroundcolor=\backclr,   % choose the background color; you must add \usepackage{color} or \usepackage{xcolor}
  basicstyle=\ttfamily\txtclr\scriptsize, % the SIZE OF THE FONTS that are used for the code
  breakatwhitespace=false,         % sets if automatic breaks should only happen at whitespace
  % breaklines=true,                 % sets automatic line breaking
  captionpos=b,                    % sets the caption-position to bottom
  commentstyle=\comclr,            % comment color
  deletekeywords={...},            % if you want to delete keywords from the given language
  escapeinside={\%*}{*)},          % if you want to add LaTeX within your code
  extendedchars=true,              % lets you use non-ASCII characters; for 8-bits encodings only, does not work with UTF-8
  frame=\frameon,                  % adds a frame around the code
  keepspaces=true,                 % keeps spaces in text, useful for keeping indentation of code (possibly needs columns=flexible)
  columns=flexible,
  otherkeywords={zip,enumerate,True,False,None,...},  %add words to be highlighted
  keywordstyle=\keyclr,            % keyword (e.g. 'print') color
  language=Python,                 % the language of the code
  upquote=true,                    % make apostrophes straight (unidirectional)
  alsoletter={<>=-+*/!},            % to avoid coloring operators when they're not
  ndkeywords={=,+,-,*,**,/,+=,*=,-=,/=,<=,>=,==,!=,<,>,... },  % operator keywords
  ndkeywordstyle=\ndkeyclr,         % style of operator keywords
  numbers=left,                    % where to put the line-numbers; possible values are (none, left, right)
  numbersep=5pt,                   % how far the line-numbers are from the code
  numberstyle=\tiny\color{mygray}, % line-number style: size, color
  rulecolor=\color{black},         % if not set, the frame-color may be changed on line-breaks within not-black text (e.g. comments (green here))
  showspaces=false,                % show spaces everywhere adding particular underscores; it overrides 'showstringspaces'
  showstringspaces=false,          % underline spaces within strings only
  showtabs=false,                  % show tabs within strings adding particular underscores
  stepnumber=1,                    % the step between two line-numbers. If it's 1, each line will be numbered
  stringstyle=\strclr,             % string color
  tabsize=4,                       % sets default tabsize
}


%Function that calls listing of specific file with custom coloring
  %first input is filename to list
  %next two inputs are start and end line numbers
    %(use 0 and large number for all lines)
\newcommand\mylisting[3]{
  \tcbinputlisting{
        listing file=#1,
        arc=0pt,
        top=0mm,
        bottom=0mm,
        left=0mm,
        right=0mm,
        boxrule=0pt,
        colback=\backclr,
        listing only,
        listing options={
                            style=mystyle,            %use custom style
                            firstline=#2,lastline=#3, %lines of code to use
                            firstnumber=#2,           %same numbering as script
                        },
        hbox
      }
      }



\usepackage{color}     %make custom colors for syntax coloring
\usepackage{listings}  %allows code listings
\usepackage{textcomp}  %allows apostrophes to be straight (unidirectional)
% \usepackage{framed}
% \usepackage{caption}
\usepackage{bm}
\usepackage{tcolorbox}        %use for code listing color definitions
\tcbuselibrary{listings}      %allow color defenitions in code listings

\captionsetup[lstlisting]{font={small,tt}}

%Custom Colors
\definecolor{mygreen}{rgb}{0,0.6,0}
\definecolor{mygray}{rgb}{0.5,0.5,0.5}
\definecolor{mymauve}{rgb}{0.58,0,0.82}

\definecolor{sublimeblack}{HTML}{272822}
\definecolor{sublimered}{HTML}{F92672}
\definecolor{sublimeblue}{HTML}{66D9EF}
\definecolor{sublimeyellow}{HTML}{E6DB74}
\definecolor{sublimegrey}{HTML}{75715E}
\definecolor{sublimegreen}{HTML}{66CC33}
\definecolor{sublimeorange}{HTML}{FD971F}
\definecolor{sublimepurple}{HTML}{AE81FF}


%White or Black background toggle
% \newcommand\whiteback{0}
\ifnum\whiteback=1%
  %White Background
  \newcommand\backclr{white}
  \newcommand\txtclr{\color{black}}
  \newcommand\comclr{\color{mygreen}}
  \newcommand\keyclr{\color{blue}}
  \newcommand\ndkeyclr{\color{sublimered}}
  \newcommand\strclr{\color{mymauve}}
  \newcommand\frameon{single} %single-frame box around code
\else
  %Black Background (like SublimeText)
  \newcommand\backclr{sublimeblack}
  \newcommand\txtclr{\color{white}}
  \newcommand\comclr{\color{sublimegrey}}
  \newcommand\keyclr{\color{sublimeblue}}
  \newcommand\ndkeyclr{\color{sublimered}}
  \newcommand\strclr{\color{sublimeyellow}}
  \newcommand\frameon{false} %no frame for black background
\fi


%Listing Style
\lstdefinestyle{mystyle}{%
  % backgroundcolor=\backclr,   % choose the background color; you must add \usepackage{color} or \usepackage{xcolor}
  basicstyle=\ttfamily\txtclr\scriptsize, % the SIZE OF THE FONTS that are used for the code
  breakatwhitespace=false,         % sets if automatic breaks should only happen at whitespace
  % breaklines=true,                 % sets automatic line breaking
  captionpos=b,                    % sets the caption-position to bottom
  commentstyle=\comclr,            % comment color
  deletekeywords={...},            % if you want to delete keywords from the given language
  escapeinside={\%*}{*)},          % if you want to add LaTeX within your code
  extendedchars=true,              % lets you use non-ASCII characters; for 8-bits encodings only, does not work with UTF-8
  frame=\frameon,                  % adds a frame around the code
  keepspaces=true,                 % keeps spaces in text, useful for keeping indentation of code (possibly needs columns=flexible)
  columns=flexible,
  otherkeywords={zip,enumerate,True,False,None,...},  %add words to be highlighted
  keywordstyle=\keyclr,            % keyword (e.g. 'print') color
  language=Python,                 % the language of the code
  upquote=true,                    % make apostrophes straight (unidirectional)
  alsoletter={<>=-+*/!},            % to avoid coloring operators when they're not
  ndkeywords={=,+,-,*,**,/,+=,*=,-=,/=,<=,>=,==,!=,<,>,... },  % operator keywords
  ndkeywordstyle=\ndkeyclr,         % style of operator keywords
  numbers=left,                    % where to put the line-numbers; possible values are (none, left, right)
  numbersep=5pt,                   % how far the line-numbers are from the code
  numberstyle=\tiny\color{mygray}, % line-number style: size, color
  rulecolor=\color{black},         % if not set, the frame-color may be changed on line-breaks within not-black text (e.g. comments (green here))
  showspaces=false,                % show spaces everywhere adding particular underscores; it overrides 'showstringspaces'
  showstringspaces=false,          % underline spaces within strings only
  showtabs=false,                  % show tabs within strings adding particular underscores
  stepnumber=1,                    % the step between two line-numbers. If it's 1, each line will be numbered
  stringstyle=\strclr,             % string color
  tabsize=4,                       % sets default tabsize
}


%Function that calls listing of specific file with custom coloring
  %first input is filename to list
  %next two inputs are start and end line numbers
    %(use 0 and large number for all lines)
\newcommand\mylisting[3]{
  \tcbinputlisting{
        listing file=#1,
        arc=0pt,
        top=0mm,
        bottom=0mm,
        left=0mm,
        right=0mm,
        boxrule=0pt,
        colback=\backclr,
        listing only,
        listing options={
                            style=mystyle,            %use custom style
                            firstline=#2,lastline=#3, %lines of code to use
                            firstnumber=#2,           %same numbering as script
                        },
        hbox
      }
      }



\usepackage{color}     %make custom colors for syntax coloring
\usepackage{listings}  %allows code listings
\usepackage{textcomp}  %allows apostrophes to be straight (unidirectional)
% \usepackage{framed}
% \usepackage{caption}
\usepackage{bm}
\usepackage{tcolorbox}        %use for code listing color definitions
\tcbuselibrary{listings}      %allow color defenitions in code listings

\captionsetup[lstlisting]{font={small,tt}}

%Custom Colors
\definecolor{mygreen}{rgb}{0,0.6,0}
\definecolor{mygray}{rgb}{0.5,0.5,0.5}
\definecolor{mymauve}{rgb}{0.58,0,0.82}

\definecolor{sublimeblack}{HTML}{272822}
\definecolor{sublimered}{HTML}{F92672}
\definecolor{sublimeblue}{HTML}{66D9EF}
\definecolor{sublimeyellow}{HTML}{E6DB74}
\definecolor{sublimegrey}{HTML}{75715E}
\definecolor{sublimegreen}{HTML}{66CC33}
\definecolor{sublimeorange}{HTML}{FD971F}
\definecolor{sublimepurple}{HTML}{AE81FF}


%White or Black background toggle
% \newcommand\whiteback{0}
\ifnum\whiteback=1%
  %White Background
  \newcommand\backclr{white}
  \newcommand\txtclr{\color{black}}
  \newcommand\comclr{\color{mygreen}}
  \newcommand\keyclr{\color{blue}}
  \newcommand\ndkeyclr{\color{sublimered}}
  \newcommand\strclr{\color{mymauve}}
  \newcommand\frameon{single} %single-frame box around code
\else
  %Black Background (like SublimeText)
  \newcommand\backclr{sublimeblack}
  \newcommand\txtclr{\color{white}}
  \newcommand\comclr{\color{sublimegrey}}
  \newcommand\keyclr{\color{sublimeblue}}
  \newcommand\ndkeyclr{\color{sublimered}}
  \newcommand\strclr{\color{sublimeyellow}}
  \newcommand\frameon{false} %no frame for black background
\fi


%Listing Style
\lstdefinestyle{mystyle}{%
  % backgroundcolor=\backclr,   % choose the background color; you must add \usepackage{color} or \usepackage{xcolor}
  basicstyle=\ttfamily\txtclr\scriptsize, % the SIZE OF THE FONTS that are used for the code
  breakatwhitespace=false,         % sets if automatic breaks should only happen at whitespace
  % breaklines=true,                 % sets automatic line breaking
  captionpos=b,                    % sets the caption-position to bottom
  commentstyle=\comclr,            % comment color
  deletekeywords={...},            % if you want to delete keywords from the given language
  escapeinside={\%*}{*)},          % if you want to add LaTeX within your code
  extendedchars=true,              % lets you use non-ASCII characters; for 8-bits encodings only, does not work with UTF-8
  frame=\frameon,                  % adds a frame around the code
  keepspaces=true,                 % keeps spaces in text, useful for keeping indentation of code (possibly needs columns=flexible)
  columns=flexible,
  otherkeywords={zip,enumerate,True,False,None,...},  %add words to be highlighted
  keywordstyle=\keyclr,            % keyword (e.g. 'print') color
  language=Python,                 % the language of the code
  upquote=true,                    % make apostrophes straight (unidirectional)
  alsoletter={<>=-+*/!},            % to avoid coloring operators when they're not
  ndkeywords={=,+,-,*,**,/,+=,*=,-=,/=,<=,>=,==,!=,<,>,... },  % operator keywords
  ndkeywordstyle=\ndkeyclr,         % style of operator keywords
  numbers=left,                    % where to put the line-numbers; possible values are (none, left, right)
  numbersep=5pt,                   % how far the line-numbers are from the code
  numberstyle=\tiny\color{mygray}, % line-number style: size, color
  rulecolor=\color{black},         % if not set, the frame-color may be changed on line-breaks within not-black text (e.g. comments (green here))
  showspaces=false,                % show spaces everywhere adding particular underscores; it overrides 'showstringspaces'
  showstringspaces=false,          % underline spaces within strings only
  showtabs=false,                  % show tabs within strings adding particular underscores
  stepnumber=1,                    % the step between two line-numbers. If it's 1, each line will be numbered
  stringstyle=\strclr,             % string color
  tabsize=4,                       % sets default tabsize
}


%Function that calls listing of specific file with custom coloring
  %first input is filename to list
  %next two inputs are start and end line numbers
    %(use 0 and large number for all lines)
\newcommand\mylisting[3]{
  \tcbinputlisting{
        listing file=#1,
        arc=0pt,
        top=0mm,
        bottom=0mm,
        left=0mm,
        right=0mm,
        boxrule=0pt,
        colback=\backclr,
        listing only,
        listing options={
                            style=mystyle,            %use custom style
                            firstline=#2,lastline=#3, %lines of code to use
                            firstnumber=#2,           %same numbering as script
                        },
        hbox
      }
      }
